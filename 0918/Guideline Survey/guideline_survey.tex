\documentclass[12pt, oneside, letterpaper]{report}
\begin{document}

\section*{Apple} ~\\


\section*{Windows} ~\\
Windows provides User Experience Interactions Guidelines which are touted to 
\begin{quote}
\begin{itemize}
\item Establish a high quality and consistency baseline for all Windows-based applications
\item Answer your specific user experience questions.
\item Make your job easier!
\end{itemize}
~\\
One feature that I noticed on every page of the documentation was the option of critiquing the page.  The page would also display how many people found the page useful.  \\
One of the design principles that stood out was the idea of \textit{reducing distractions} The guidelines exemplify this. In comparison to the Apple Guidelines, Windows Guidelines are presented in a clean manner, without clutter. Even the size of the page is smaller than its apple counterpart. \\
I looked at the desktop page, which 37 of 61 people rated helpful.  



\end{quote}

\section*{Linux} ~\\
Unlike the first two operating systems, Linux's Ubuntu does not have a definite set of human interface guidelines. However, Ubuntu does provide community documentation and a basic philosophy as well as a wikibook link. \\
The word \textit{Ubuntu} itself finds its roots in African culture.  From Wikipedia, Ubuntu is defined as \textit{"I am what I am because of who we all are."}  Such philosophy is reflected in the community documentation of the Ubuntu system.  The community documentations, hosted by Ubuntu, tout that they are not the "official documentation," but rather a collection of "Howtos, Tips, Tricks, and Hacks." Though the document has been through many edits and many hands, it maintains consistency in its writing and formatting.   
 It provides an easy to follow click-through experience that leads to How-To's. Though these are fairly easy to follow, it is apparent that the documents are geared towards those who are more than proficient with the computer.  The document is is camel case, and each has a how to on the command line.  \\
These Ubuntu promoted boards reflect their philosophy which is based firmly upon the ideal that everything should be free and open source.  Ubuntu remains very consistent in this, providing an entire library of programs that are open source.  The Ubuntu Software Center embodies this and provides a GUI for all of the open source software avaliable on the Ubuntu system.  \\
Additionally, they have a help documentation that provides detailed instructions on accessing settings for all parts of the Ubuntu system. The help menu itself is a little confusing, and the documentation for everything provides no graphics, and only words. To find the instructions it took 3 clicks from the starting help page, and the instructions were enumerated blocks of text. \\
\end{document}

