\documentclass[12pt, onesided, letterpaper]{report}
\begin{document}
\title{Usability Metrics in the Desktops of Various OS}
\date{18 September 2012}
\author{Kaitlyn Higa \and Lisa Rosenbaum}
\maketitle


\section*{Introduction}
\paragraph*{Interaction Design}~\\

In the second chapter of their text \emph{Designing the User Interface}, Shneiderman and Plaisant discuss ``\emph{standards}'' to follow while creating systems.  The chapter separates these rules into Guidelines, specific definitions distributed and adhered to by companies and organizations and Principles, broader generalizations that are more ``fundamental.''  \\

For this report, personalization tasks on the Mac, Windows, and Linux systems were pitted against each other to test their usability.  Each provides a basic philosophy or guidelines for developers to follow.  Further discussion of the guidelines can be found in the \textit{/guideline survey/} section of the repository.  
\paragraph*{Usability Metrics} ~\\

In order to create a concrete, non-subjective method of rating the usability of systems, a set of 5 usability metrics were created.  These assign qualitative values in order to rate systems numerically. 
\begin{enumerate}
\item \textbf{Learnability}: the amount of time that it takes to learn the system.  
\item \textbf{Memorability}: the amount of time that it takes to recall how to use a system after a period of time has passed since learning.  
\item \textbf{Efficiency}: the amount of time it takes to complete a task using the system.
\item \textbf{Errors}: the amount of errors made while completing a task while using the system.
\item \textbf{Satisfaction}: the amount of satisfaction gained while using the system. 
\end{enumerate}

\paragraph*{Usability Metrics in context of this assignment}~\\
We are comparing the interaction designs of Mac, Windows and Linux operating systems.  Our focus will be the desktop.  We will be timing how long it takes to change the desktop background, change the screen resolution, and take a screen shot.  We will be looking at three of the four usability metrics: Learnability, Efficiency, and Satisfaction.  \\
~\\
Learnability is how fast a person who knows nothing about a system can accomplish something with that system.  How fast can the user figure out how to get to the change desktop menu?  How easy is it for the user to figure out how to change the screen resolution?  How fast can the user learn how to take a screen shot?  We are comparing how well each operating system helps the user learn how to accomplish the three tasks.  We will be asking some users to complete the tasks on systems they are not familiar with.\\

~\\
Efficiency is how fast a person who knows how to use a system well can accomplish a certain task.  Which system allows the user to change the desktop background, change the screen resolution and take a screen shot the fasted?  We will compare how fast each operating system allows the user to complete the tasks.  We will be timing users who are skilled with each system.\\
~\\
Satisfaction is how happy a user is with a system.  We are recording each person's opinion after completing each task in order to see which operating system is the most pleasing for the user.  How pleased was the user with the process of changing the background, changing the screen resolution and taking a screen shot?\\

\pagebreak

\section*{Procedure}

\paragraph*{3 Concrete Tests} ~\\
In order to test system usability, the following three tests were conceived:
\begin{enumerate}
\item Saving a jpg via the Chrome Browser to the computer and setting it as the background
\item Changing the Screen Resolution
\item Taking a screenshot.
\end{enumerate}

\paragraph*{Procedure} ~\\
For each subject, we prepared, on a Linux, Windows, and Mac, the Chrome Browser with a picture that had never been saved to the computer before and closed all superfluous windows. The system settings icon remained in the taskbar of every system.  The subjects were briefed on the tests as follows:
\begin{enumerate}
\item \textbf{Background Changing}: You must save this (indicates) picture and set it as your desktop background.  Feel free to open any window or use any sort of help dialog or search engine.
\item \textbf{Screen Resolution Change}: Alter the screen resolution from the current set-up to any other setting.
\item \textbf{Taking a Screen Shot}: Take a screenshot of the desktop and save it.
\end{enumerate}
We first tested the subject on the foreign system and then on their usual systems. The times were recorded on a Samsung Galaxy S3 default clock/stopwatch.  Milliseconds were not recorded.  Satisfaction was rated on a scale of 1-10, with 1 being the least satisfaction and 10 being the most satisfaction.
\pagebreak

\section*{Data Reports}
Our first subject was a male junior, who works primarily on Apple, has had some Windows experience, and no experience on Linux. His main system was Apple and new system was Linux.
\begin{center}
    \begin{tabular}{ | l | l | l |}
    \hline
    Task &Main System & New System  \\ \hline
    Changing Background & 0:07 & 1:16  \\ \hline
    Changing Screen Resolution & 0:09 & 0:39  \\ \hline
    Screen Shot & 1:07 & 0:02 \\
    \hline
    \end{tabular}
\end{center}
The subject was very agitated when setting the desktop on the Linux.  He had also forgotten the shortcut on the mac and was trying multiple combinations and opened up music players and browsers in the meanwhile. 
He rated the Linux a 6/10 satisfaction and the Apple an 7/10.

Our second subject was a female junior, who works primarily on Apple, has had some Windows experience, and no experience on Linux.  Her main system was Apple and new system was Linux.
\begin{center}
    \begin{tabular}{ | l | l | l |}
    \hline
    Task &Main System & New System  \\ \hline
    Changing Background & 0:04 & 1:12  \\ \hline
    Changing Screen Resolution & 0:16 & 0:45  \\ \hline
    Screen Shot & 0:02 & 0:01 \\
    \hline
    \end{tabular}
\end{center}
She rated the Linux a 4.5/10 satisfaction and the Apple an 8.5/10.

Our second subject was a female junior, who works primarily on Apple, has had some Windows experience, and little experience on Linux. Her main system was Apple and new system was Linux.
\begin{center}
    \begin{tabular}{ | l | l | l |}
    \hline
    Task &Main System & New System  \\ \hline
    Changing Background & 0:30 & 2:41  \\ \hline
    Changing Screen Resolution & 0:16 & 1:12  \\ \hline
    Screen Shot & 0:03 & 0:01 \\
    \hline
    \end{tabular}
\end{center}
She rated the Linux a 6.5/10 satisfaction and the Apple an 7.5/10
\pagebreak
\section*{Reflections}
\textit{These reflections reflect those of Kaitlyn Higa.  The same report, but with Lisa Rosenbaum's reflections are contained within the other file of this folder 0918\_l.pdf}

Overall, these results illustrate the usability of both systems.  Revealed in discussion with the subjects, they all said that they purchased their Apple systems reasoning that they were easy to learn and use and wouldn't have many problems.  Though we did not test new Apple users, the three veteran Apple users said that they were confident in their systems and were comfortable in knowing how to use their computer. When questioned about the Linux system, they estimated it would take more time to learn it than it had taken them to learn their Apple systems.

One big difference between the two systems, however is the ease of screenshots.  The Linux box had a dedicated screen shot button (prt sc) where as the Apple has a handful of shortcuts (command-shift-4-space.)  Though in the scheme of things, screenshots are not the most essential thing, it breaks with the guidelines of simplicity.
\end{document}
