\documentclass[12pt, onesided, letterpaper]{report}
\begin{document}
\title{Usability Metrics in the Desktops of Various OS}
\date{18 September 2012}
\author{Kaitlyn Higa \and Lisa Rosenbaum}
\maketitle


\section*{Introduction}
\paragraph*{Interaction Design}~\\
In the second chapter of their text \emph{Designing the User Interface}, Shneiderman and Plaisant discuss "\emph{standards}" to follow while creating systems.  The chapter separates these rules into Guidelines, specific definitions distributed and adhered to by companies and organizations and Principles, broader generalizations that are more "fundamental."  \\
For this report, personalization tasks on the Mac, Windows, and Linux systems were pitted against each other to test their usability.  Upon examination of the user guidelines, it would appear that 
\paragraph*{Usability Metrics} ~\\
In order to create a concrete, non-subjective method of rating the usability of systems, a set of 5 usability metrics were created.  These assign qualitative values to ;p;p;
\begin{enumerate}
\item \textbf{Learnability}: the amount of time that it takes to learn the system.  
\item \textbf{Memorability}: the amount of time that it takes to recall how to use a system after a period of time has passed since learning.  
\item \textbf{Efficiency}: the amount of time it takes to complete a task using the system.
\item \textbf{Errors}: the amount of errors made while completing a task while using the system.
\item \textbf{Satisfaction}: the amount of satisfaction gained while using the system. 
\end{enumerate}

\paragraph*{Usability Meterics in context of this assignment}~\\
 

\pagebreak
\pagebreak
\section*{Procedure}

\paragraph*{3 Concrete Tests}

\paragraph*{Procedure}


\pagebreak

\section*{Data Reports}


\section*{Reflections}


\end{document}
